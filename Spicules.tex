\chapter[Spicules at the limb]{Spicules at the limb \label{ch:spicules}\Large{\protect\footnote{Contents from this Chapter have been partially published as \cite{2007A&A...472L..51S}}}}

Spicules, known for more than 130 years \citep[see the hand drawings by][]{secchi1877}, represent a prominent example of the dynamic
chromosphere. We refer the reader to reviews by \citet{beckers68,beckers72} and to the paper by \citet{wilhelm00} on UV properties. According to these works, spicules are seen at and outside the limb of the Sun as thin, elongated features. They develop speeds, measured from both proper motion and Doppler shifts, of 10--30 km\,s$^{-1}$ and reach
heights of 5--9 Mm on average, during their lifetimes of 3--15 minutes. Recent observations from HINODE \citep{2007SoPh..243....3K} and own results presented below in Sec. \ref{sec:limb:ha} have changed the traditional picture. Some spicules live for only few seconds, and spicules may be much more inclined with respect to the vertical than adopted hitherto.  

As pointed out by \citet{sterling00}, a key impediment to develop a satisfactory understanding has been the lack of reliable observational data. 
Many theoretical models have been proposed to understand the nature of spicules, using a wide variety of motion triggers and driving mechanisms. 
In this Chapter we focus on the \ion{He}{i} 10830\,\AA\ triplet emission line (see Sec. \ref{sec:limb:he}), using recent technical improvements in observational facilities, and on the results from the limb observations in H$\alpha$.

\section{Spicule emission profiles observed in \ion{He}{i} 10830 \AA\label{spicules}}

The energy levels that take part in the   \ion{He}{i} 10830 \AA\, triplet are basically populated via
an ionization-recombination process \citep{1994isp..book...35A}. The EUV coronal irradiation (CI) at 
wavelengths $\lambda<504$~\AA\ ionizes the neutral helium, and subsequent recombinations of singly ionized helium with free electrons lead to an overpopulation of all ortho-helium levels. Alternative theories suggest other mechanisms that might also contribute to the formation of the helium lines relying on the collisional excitation of the electrons in regions with higher temperature \citep[e.g.,][]{1997ApJ...489..375A}.  We are able to provide observational evidence of the link between the corona and the infrared emission of this line, in the frame of the current theoretical models of the solar atmosphere.
 
\citet{Centeno06} modelled the ionisation and recombination processes using various   
 amounts of CI, non-LTE radiative transfer, and different atmospheric
 models {  \citep[see also][]{cente07}}.
 They have simulated limb observations for different heights, obtaining
 synthetic emission profiles in spherically symmetric models of the solar atmosphere. One conclusion of their  study is that the ratio of intensities $({\cal R}=I_{\rm blue}/I_{\rm red})$ of the `blue' to the `red' components of the \ion{He}{i} 10830~\AA\ emission
 is a very good candidate for diagnosing the CI. The population of the metastable level depends on optical thickness, whose variation with height governs the change in the ratio $\cal R$ as a function of the distance to the limb.
 
\citet{truj05} measured the four Stokes parameters of quiet-Sun chromospheric spicules and could show evidence of the Hanle effect by the action of inclined magnetic fields with an average strength of the order of 10 G. They modelled the \ion{He}{i} 10830~\AA\
profiles assuming the medium along the integrated line of sight as a slab of constant properties and with its optical thickness as a free parameter. \citet{truj05} showed that the observed intensity profiles and their ensuing $\cal R$ values can be reproduced by choosing an optical thickness significantly larger than unity. \citet{Centeno06} demonstrated that this optical thickness is related to the coronal irradiance (through the ratio $\cal R$), thus providing a physical meaning to the free parameter in the slab model (see also Centeno et al. 2007).
\begin{figure}[t]
\center \includegraphics[width=\textwidth]{../figures/perfiles.pdf} 
\caption{Measured \ion{He}{i} 10830~\AA\ emission profiles for increasing
  distances to the solar limb, scanning a broad range of the height extension of the 
  spicules. Each profile is the average of the 312 pixels along the slit (which was always kept parallel to the limb).}  
\label{fig:spe}
\end{figure}

\subsection{Observational  %\ion{He}{i}
 intensity profiles and intensity ratio}

We present novel observations showing the spectral emission of \ion{He}{i} 10830~\AA\ and its dependence on the height of the
spicules above a quiet region. We compare the deduced observational $\cal R$ with that obtained from detailed non-LTE numerical calculations using available atmospheric profiles. 

These data correspond to the data set described in Table \ref{table:obs:tip}. After the standard reduction process (Sec. \ref{tip:reduc} ) we obtain 21 intensity profiles above the infrared limb, with a step size of $0\farcs35$. Figure~\ref{fig:spe} shows the emission profiles of the \ion{He}{i} 10830~\AA\, (after the reduction process) for different heights above the limb. 
Figure~\ref{fig:3d} illustrates this in three dimensions, as a function of
wavelength and the distance to the solar limb, clearly showing 
a change in the intensity ratio of the blue and red components of the 
multiplet $({\cal R}=I_{\rm blue}/I_{\rm red})$ with height. 
%

\begin{figure}[t]
\center \includegraphics[width=\textwidth]{../figures/3d-tip.pdf} 
\caption{3D representation of the measured \ion{He}{i} 10830~\AA\ emission
  profiles for increasing distances to the solar visible limb. Note that the  x-axis is
  wavelength, the y-axis the height above solar limb and the z-axis the intensity
  normalised to the maximum emission in the line center of the red component.}
\label{fig:3d}
\end{figure}


For the calculation of $\cal{R}$ we need to determine the amplitudes of the 
blue and red components of the emission profile (as shown in Fig.  
\ref{fig:ajuste}).

To determine the core wavelength of the red 
component of the triplet we fitted a Gaussian profile to its core, in a 1.3 \AA\ 
range around the maximum. After symmetrising the observed profile around this 
maximum, using the values on the red side of the red component, we fitted another Gaussian 
function to the resulting symmetric profile. Subtraction of the fitted 
symmetric profile from the data leaves the emission profile of the blue 
component, which was also approximated by a Gaussian to determine its central wavelength.  Our tests trying to fit directly both profiles using two Gaussians failed in a number of cases, probably due to the following reasons: (a) the red component is in fact the result of two blended lines, (b) the much weaker blue component was almost hidden in the broadened red component, and (c) the presence of noise. Our technique determines first the red component and then, after subtraction of the fitted profile, the blue one.



We have thus separated the helium emissions into their red and blue
components  assuming only that both are present and that they are both symmetric. We can now measure their widths and intensities and also check that the line core positions coincide with the theoretical ones. After the fitting process the residuals
between measured and observed profiles were small, the largest errors occurring
in the determination of the core intensities of the red line. This happens because the red component consists of two blended lines (with a separation of 0.09 \AA), a fact that flattens the emission profile near the core as opposed to a more peaked Gaussian function. Nevertheless, the differences between fitted profiles
and data are only significant in the red core and are always lower than
+0.08 of the maximum normalised intensity, with a mean difference of $\sim$0.02. To avoid systematic errors, we used the observational values for the center of
the red component when calculating $\cal R$. 


\begin{figure}[t]
\center \includegraphics[width=\textwidth]{../figures/he05red.pdf} 
\caption{Determination of the blue and red components of the \ion{He}{i} 
10830~\AA\ triplet from the observed emission profiles. In this example the
 slit was placed at 1\farcs4 off the solar visible limb. See text for details. 
  The solid line {  represents} the average emission profile. The dotted line is the Gaussian fit to the symmetrised red
  component. Subtraction of this from the observed profile leaves the blue
  component, which is also fitted by a Gaussian profile (thin solid line). 
The sum of both Gaussians (dashed line)  gives the fit to the 
observed profile.}
\label{fig:ajuste}
\end{figure}

\subsection{Results}% on the He IR triplet intensity ratio}
The chromospheric temperature and density are too low  to populate the
ortho-helium levels via collisions \citep{1994isp..book...35A}. The EUV irradiation from
the corona (CI) ionises the para-helium, and  
the subsequent recombinations lead to an overpopulation of all the 
ortho-helium levels, in particular those involved in the 10830~\AA\
transitions. \citet{Centeno06} and Centeno et al. (2007) have modelled the off-the-limb emission profiles and concluded that the ratio $\cal{R}$\,=\,$I_{blue}/I_{red}$ is a function of the height and a direct
tracer of the amount of CI. Here we compare the results from the theoretical modelling with observations.

\citet{truj05} {modelled their spectropolarimetric observations assuming a slab with constant physical properties with a given optical thickness}. In the optically thin regime $\cal R$\,$\sim$\,0.12, which is the ratio of the relative oscillator strengths of the triplet. As the optical
thickness (at the line-center of the red blended component) grows, this ratio also increases 
until it reaches a saturation value slightly larger than 1 for $\tau\sim10$. (This type of calculation can be done and improved as explained in Trujillo Bueno \& Asensio Ramos 2007). To reproduce the observed emission profile \citet{truj05} had to choose $\tau \sim 3$. Interestingly, {  the values of $\tau$ yielded by this modelling strategy are consistent} with the more realistic 
approach of Centeno (2006), where non-LTE radiative transfer calculations in 
semi-empirical models of the solar atmosphere are presented, using spherical 
geometry and taking into account the ionising coronal irradiation.
With our data we are able to test such theoretical calculations by comparing 
the measured values of $\cal R$ with those resulting from various chromospheric
models. This way we may eventually trace the amount of CI inciding on the 
spicules. The analysis described above yielded the 
values of $\cal R$ for the observed profiles. The resulting dependence on 
the distance to the solar
limb, for each pixel along the slit and each position of the slit above the
limb, are presented in Fig.~\ref{fig:ratios}. The solid black line gives the average value of $\cal R$.

\begin{figure}[t]
\center \includegraphics[width=\textwidth]{../figures/he05rel.pdf}
%\vspace{-0.5cm} 
\caption{Measured ratio $\cal R$\,=\,$I_{blue}/I_{red}$ as a function of
  distance to the solar limb. Thin lines come from
  each pixel along the slit. The thick solid line represents the average and the dashed line the value of the optically thin regime.} 
\label{fig:ratios}
\end{figure}

The dependence of $\cal R$ with height can be understood in a qualitative way as
follows: In the outer layers of the chromosphere the density is so low 
that the transitions occur in the optically thin regime. 
With decreasing altitude the ratio $\cal R$ increases (proportionally with 
density) until a maximum optical thickness is reached.  At even lower 
layers, although the density still continues to rise,  the extinction of the coronal
irradiance leads to a reduction in the number of ionizations, which results 
in a decrease of the optical thickness in the core wavelength of the red 
component, {  and thus in a decrease of $\cal R$.}

%\section{Comparison modelling vs. observations}

For a quantitative comparison with theoretical modelling we have 
used the results from \citet{Centeno06} and \cite{cente07} where they calculated the ratios $\cal R$ for different
standard model atmospheres: FAL-C and FAL-P \citep{fontenla91} and 
FAL-X \citep{avrett95}. The FAL-C and FAL-X models may be considered as illustrative of the thermal conditions in the quiet Sun, while the FAL-P model of a plage region. The FAL-X model has a relatively cool atmosphere in order to explain the molecular CO absorption at 4.6~$\mu$.

The comparison is shown in Fig.~\ref{fig:comp}. We notice that the modelled
height variations of $\cal{R}$ agree only in a qualitative manner with what is 
found in our observations. However, the calculations from different models of 
the solar atmosphere are unable to reproduce the measured ratio. Higher values of the coronal irradiance lead to an increase of the optical thickness (at the line centers of the \ion{He}{i} multiplet) and an upward shift in the run of $\cal{R}$ vs. height. Yet the shape of the height dependence is mainly given by the atmospheric density profile and the attenuation of the ionising radiation as it reaches the lower layers of the chromosphere.
It is also clear from Fig.~\ref{fig:comp} that the models do not extend high enough. 

\begin{figure}[t]
\hspace{-0.5cm}\includegraphics[width=\textwidth]{../figures/he05teo.pdf} 
\caption{Observed (average) vs. theoretical variation of the ratio $\cal R$$ =
  I_{blue}/I_{red}$ 
  with height.}
\label{fig:comp}
\end{figure}

\subsection{Conclusions\label{conclusion}}
The theoretical behaviour of the ratio $\cal R$ agrees
qualitatively with observations. Yet, a quantitative comparison shows poor
agreement. Also, the simulated ratios are highly model dependent. 
As already explained, {  the failure to reproduce the observed profiles is very likely due to the density stratification not being adequate for spicule modelling and to the limited vertical extension of the atmospheric models.}
Modelling of the intensity ratio $\cal R$ in the
\ion{He}{i} infrared triplet should account for the fact that the solar
chromosphere is inhomogeneous on small scales and that the spicules are
small-scale intrusions of chromospheric matter into the hot corona.
\begin{comment}
New data of spicule regions near the poles and the equator, below coronal 
holes or coronal active regions should help us to understand the detailed 
behaviour of the \ion{He}{i} 10830~\AA\ lines.
In further work, we will extend this study to the full Stokes vector, in order to see  
the variation of the linear polarization - or even the variation of the Hanle effect - with height. {  It would also be interesting to use the most recent models of active region fibrils and spicules \citep[e.g.,][]{hegg07} in order to see whether or not they agree with our observations.} Future models of the solar chromosphere should be constrained by the observational evidences presented here.
\end{comment}

%\section{Hanle effect\label{hanle}}
%\textbf{Should I stick to the letter? then this section should be removed.}

\section{High resolution imaging of spicules\label{sec:limb:ha}}

\cite{2007arXiv0710.2934D} recently published high resolution observations of spicules with the Solar Optical Telescope on board Hinode  \citep{2007SoPh..243....3K} in the \ion{Ca}{II} H line at 3968 \AA. They find at least two types of spicules that dominate the structure of the magnetic solar chromosphere: Type I with 3-7 minute timescales that correspond to the hitherto known spicules, and the new Type II spicules, developing in  $\sim10$ s, and lasting 10-150 s. These are also very thin, with widths down to the spatial resolution (120 km).  Also, \cite{truj05} used spectropolarimetric observations and a theoretical modeling accounting for radiative transfer effects. They find that the magnetic field in spicules is aligned with the visible structures and measure field strengths of up to 40 G with an inclination of 35$^{\circ}$ with respect to he local vertical. 


This Section compares these observational and theoretical properties with our high spatial resolution observations with the G-FPI. We use the dataset ``limb''. It consists of several consecutive H$\alpha$ scans with a field of view that includes the limb and a region outside the disc up to a height where no emission from spicules in the H$\alpha$ core is observed. The \emph{seeing} conditions were extremely good during the observations and the AO system could lock on a nearby facula. After usual dark subtraction and flat fielding we have used the BD method (see Sec. \ref{momfbd}) to achieve highest spatial resolution. We were observing the limb near both poles. In Figures \ref{fig:spicules} and \ref{fig:spicules2} we present some examples of the reduced data. 

Our time series of four minutes duration already shows a wide range of dynamics. We observe both long lasting spicules and fast evolving phenomena. Measuring the inclination of the projected spicules to the local vertical we find angles up to 30\degr\ for the north pole, as it has been reported \citep{beckers68} . The projected  height above the limb varies from the wings to the core, from 2770 to 3750 km at $\pm 0.5$ and $\pm 1$\AA\, respectively. Near the south pole we find, however, much stronger emission and higher inclinations. The maximum angle is close to 70\degr\ from the local vertical, while the maximum height reaches up to 8250 km. We also find one horizontal fibril/spicule, as well as the presence of kinks or bends in some spicules. The width of single resolved spicules varies from a maximum width of 1\,000 km at the spicule footing to a minimum size of 250 km, almost down to the resolution limit of the images, both in faint spicules and in others with strong emission. We also can retrieve the spectral profile at each pixel. 

Figure \ref{fig:spicules} demonstrates an important contribution to the understanding of spicules. It solves the long-standing question about the counterparts of spicules on the disc \citep{1992A&A...264..236G}. The first and last four filtergrams of the scan across H$\alpha$ in this figure show that spicules outside the limb continue as dark fibrils inside the disc.

In Fig. \ref{fig:spiprofima} we show the mean variation from the disc to the limb of the intensity in H$\alpha$  around the north pole. Further, Fig. \ref{fig:spiprof} presents mean intensity variations from the disc to the limb for several wavelengths around the H$\alpha$ line center. The emission at the line center is almost constant from the disc up to a height of around 5\arcsec\ above the limb, where the intensity starts to decrease.


\begin{figure}
\center \includegraphics[width=\textwidth]{../figures/espicules.pdf} 
\caption{Reconstructed narrow-band scan observed near the solar north pole. The size of each image is $56\farcs1\times19\farcs1$. The wavelength of the filtergrams increases by $0.1$\AA\, from top left to bottom right row by row. The third image in the third row is closest to the center of the mean line profile. The images have been rotated to have the limb parallel to the horizontal axis.}  
\label{fig:spicules}
\end{figure}
\begin{sidewaysfigure}
\center \includegraphics[width=\textwidth]{../figures/espicules2B.pdf} 
\caption{Limb H$\alpha$ 2D filtergram at $\lambda_{0} +1.1$ \AA\, near the south pole, where a coronal hole was present. This region shows much stronger emission and more variation of spicule width, height and inclination as Fig. \ref{fig:spicules}. A background of thin vertical spicules can be seen overlaid with wider and more inclined spicules, including nearly horizontal jets. Some of the spicules appear to be bent and show internal structure such as splitting into parallel jets. The maximum projected height above the limb is $\approx 8\,250$~km, while the mean height at this wavelength is $\approx$3700 km. The image has been rotated to show a horizontal limb in the presentation.}  
\label{fig:spicules2}
\end{sidewaysfigure}
\begin{figure}
\center 
\includegraphics[width=\textwidth]{../figures/limbha.pdf} 
%\includegraphics[width=\textwidth]{../figures/limbha2.pdf}
\caption{Image representation of the mean measured spicular profiles from image \ref{fig:spicules}. the x-axis is the height above limb, while the y-axis is the wavelength around H$\alpha$ line center (black horizontal line). Horizontal cuts at $\lambda-\lambda_{0}=[0,\pm 0.5, \pm 1]$\AA\, are shown in Fig. \ref{fig:spiprof}.} 
\label{fig:spiprofima}
\end{figure}

\begin{figure}
\center \includegraphics[width=\textwidth]{../figures/profha.pdf} 
%\includegraphics[width=0.4\textwidth]{../figures/profha2.pdf} 
\caption{Average over $11\farcs2$ of H$\alpha$ intensity profiles inside and outside the limb, for several line positions, observed near the solar north pole. \emph{Dotted line}: broadband intensity at 6300 \AA, the inflection point defines the position of the solar limb; \emph{green thick line}: intensity at H$\alpha$ line center, which is nearly constant till $\approx$5\arcsec\ above the limb and then decreases outwards; \emph{dashed lines}: intensities at $-0.5$ \AA\,(blue) and $+0.5$ \AA\,(red) off line center, with height of spicular visibility decreasing at $\approx3\farcs7$; \emph{solid lines}: intensities at $\pm1$\AA\, off line center. It is seen that the H$\alpha$ line turns from absorption inside the limb into an emission line (line intensities higher than the continuum intensity) above the limb.}  
\label{fig:spiprof}
\end{figure}




